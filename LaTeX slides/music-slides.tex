\documentclass[usenames,dvipsnames]{beamer}
\usepackage{amsmath}
\usepackage{mathtools}
\usepackage{bussproofs}
\usetheme{metropolis}
\usefonttheme[onlymath]{serif}
\title{Harmony and Pitch Groups}
\date{September 18th, 2020}
\author{Ali Taqi}
\institute{Reed Student Colloquium}

\newcommand{\N}{\mathbb{N}}
\newcommand{\Q}{\mathbb{Q}}

\begin{document}

  \maketitle
  \section{Origins: The Integers}
  %%Add quote: "And first, God made the integers."
  \begin{frame}{Integers}
  The integers are a set of numbers, defined as follows: \newline
  $$\mathbb{Z} = \{...,-2,-1,0,1,2,...\}$$
  \newline
  However, the set of numbers we are more interested in is the natural numbers. \newline
  $$\mathbb{N} = \{1,2,3,...\}$$
  \end{frame}

  \begin{frame}{Harmonics}
  One main element of music is pitch. Pitch is a resonance at a particular frequency, say $f$. \newline
  Example: $\lambda = $A4 $\Rightarrow f_\lambda = 440Hz$ \newline
  \end{frame}
    \begin{frame}{Harmonics}
  In nature, there exists something called the Harmonic series. From physics, we know that a resonant body doesn't really ever resonate at a pure frequency. Instead, the frequency we perceive it as is called the fundamental frequency. 
  
  The harmonic series is defined as the product of some fundamental frequency and the series of natural numbers. Define the harmonic series of a given frequency $f$ to be the set:\newline 
  $$\text{Harm}[f] = \{nf : n \in \mathbb{N}\}$$
  \end{frame}
      \begin{frame}{Harmonics: Example}
      Let us dissect this notation. Let us take the same note as before, an octave lower. That is, let $\lambda = \space$ A3. So, $f_\lambda = 220Hz$. Now, recall that the natural numbers are defined as the set $\mathbb{N} = \{1,2,3,...\}$. From our definition of the set of harmonics, we obtain:
  \begin{align*}
  \text{Harm}[f_\lambda] &= \{(1)\cdot220Hz, (2)\cdot220Hz, (3)\cdot220Hz, ...\} \\
  &= \{220Hz, 440Hz, 660Hz, ...\}
  \end{align*}
      \end{frame}

\begin{frame}{Consonance and the Primes}
The prime numbers is a subset of $\mathbb{N}$, and we denote it the set $\mathbb{P} \subset \mathbb{N}$.

We can start defining some actual musical objects now. Take the subset $A  = \{1,3,5\} \subset \N$. Then the harmonic subset with respect to $A$ can be defined as: 
\begin{align*}
\text{Harm}_A [f] &= \{af : a \in A\} \\
&=\{f,3f,5f\}
\end{align*}

\end{frame}
  
\end{document}